\section{Obtendo Ajuda}

Aprenda a usar bem os arquivos da Ajuda ("Help"). Muitas vezes, ao final de cada página da Ajuda, há exemplos úteis sobre o tópico em questão.  Role a página para baixo para conferi-los, mesmo que (ou especialmente se) você não tenha entendido completamente as explicações do texto. Você pode rodar os exemplos diretamente de página de Ajuda do SuperCollider, ou você pode copiar e colar o código em uma nova janela para fuçar e experimentar mais com ele. 

Selecione qualquer classe ou método válidos no seu código de SuperCollider (um duplo-clique na palavra irá selecioná-la) e aperte [ctrl+D] para abrir o arquivo de Help correspondente. Se você selecionar o nome de uma classe (por exemplo, \texttt{MouseX}), você será direcionado para o arquivo de Ajuda da classe.\footnote{Atenção: O SuperCollider vai mostrar em azul qualquer palavra que começa com uma letra maiúscula. Isso significa que a cor azul \emph{não garante} que a palavra esteja livre de erros: por exemplo, se você digitar Sinosc (com o "o" minúsculo incorreto), ainda assim a palavra aparece em azul.} Se você selecionar um método, você será direcionado a uma lista de todas as classes que entendem aquele método (por exemplo, peça ajuda do método \texttt{scramble}).

Outras maneiras de explorar os arquivos de Ajuda do SuperCollider são os links "Browse" ("Navegar") e "Search" ("Buscar"). Use o Browse para navegar os arquivos por categorias e o Search para pesquisar palavras em todos os arquivos de Ajuda.
Nota importante sobre o Browser de Ajuda no IDE do SuperCollider:

\begin{itemize}
\item Use o campo superior direito (onde se lê "Find…") para procurar palavras específicas \emph{dentro do arquivo de Help que está aberto} (da mesma maneira que você usaria um "find" para localizar algo em um website ou num documento de texto);
\item Use o link "Search" (à direita de "Browse") para procurar texto \emph{em todos os arquivos de Ajuda}.
\end{itemize}

Quando você abre o primeiro parêntese para adicionar argumentos de um método específico, o SC mostra uma pequena "dica de ajuda" para mostrar quais são os argumentos esperados. Por exemplo, digite o início de uma linha como mostrado na figura \ref{fig:tooltip}. Logo depois de abrir o primeiro parêntese, a dica mostra que os argumentos para um \texttt{SinOsc.ar} são \texttt{freq}, \texttt{phase}, \texttt{mul} e \texttt{add}. Também aparecem os valores padrão. Esta é exatamente a mesma informação que você obteria no arquivo de Ajuda do \texttt{SinOsc}. Se a dica de ajuda desapareceu, você pode trazê-la de volta com [ctrl+Shift+Espaço].

\begin{figure}[h]
\centerline{\framebox{
	\includegraphics[scale=0.3]{fig-help-tooltip-crop.png}}}
\caption{Informações úteis mostradas conforme você vai digitando.}
\label{fig:tooltip}
\end{figure}

Outro atalho: se você quiser explicitamente nomear seus argumentos (como \texttt{SinOsc.ar(freq: 890)}), experimente apertar a tecla Tab logo depois de abrir o parêntese. O SC vai autocompletar para você com o nome do argumento correto, na ordem, à medida que você digita (aperte Tab depois da vírgula para os nomes dos argumentos subsequentes).

\bigskip
\todo[inline, color=green!40]{ 
DICA: Crie uma pasta com seus próprios "arquivos pessoais de ajuda". Sempre que você descobrir um novo truque ou aprender um novo objeto, escreva um exemplo simples com explicações em suas próprias palavras e salve-o para o futuro. Pode apostar que esses arquivos pessoais vão ser muito úteis depois de um mês ou um ano, quando você precisar lembrar como funciona tal ou qual objeto ou mensagem.
}
\bigskip

Os arquivos de Ajuda podem ser também consultados online: \url{http://doc.sccode.org/}.
