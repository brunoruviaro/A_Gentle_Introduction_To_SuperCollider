\section{Comentários}

Todo o texto no seu código que aparece em vermelho é um \emph{comment}. Se você é novo em linguagens de programação, comentários são uma forma bastante útil de documentar o seu código, tanto para você mesmo quanto para outros que tenham de lê-lo depois. Qualquer linha que começa com uma barra dupla é um comentário. Você pode escrever comentários logo depois de uma linha válida de código, pois a parte do comentário será ignorada quando você rodar. No SC, usamos um ponto-e-vírgula para indicar o fim de um enunciado válido.

\begin{lstlisting}[style=SuperCollider-IDE, basicstyle=\scttfamily\footnotesize]
2 + 5 + 10 - 5; // apenas fazendo contas

rrand(10, 20); // gerar um numero randômico entre 10 e 20
\end{lstlisting}

Você pode rodar uma linha mesmo que o seu cursor estiver no meio de um comentário depois desta linha. A parte do comentário é ignorada. Os próximos dois parágrafos serão escritos como “comentários” apenas como exemplo.


 
\begin{lstlisting}[style=SuperCollider-IDE, basicstyle=\scttfamily\footnotesize]
// Você pode rapidamente transformar uma linha de código em comentário usando o atalho [ctrl+/].
“Algum código de SC aqui…”.postln;
2 + 2;


//Se você escrever um comentário realmente longo, pode haver uma quebra em seu texto que parece uma nova linha que *não* começa com uma barra dupla. No entanto, isso ainda conta como uma única linha de comentário.

/* Use “barra + asterisco” para começar um comentário longo com diversas linhas. Feche o trecho de comentário com “asterisco + barra”. O atalho mencionado anteriormente também funciona para grandes trechos: simplesmente selecione as linhas de código que você quer “comentar” e pressione [ctrl+/]. O mesmo serve para des-comentar.*/
\end{lstlisting}
