\section{Comentários}

Todo o texto que aparece em vermelho no seu código é um \emph{comentário}. Se você é novo em linguagens de programação, comentários são bastante úteis para documentar o seu código, tanto para você mesmo, quanto para outros que tenham de lê-lo depois. Qualquer linha começando com uma barra dupla é um comentário. Você pode escrever comentários logo depois de uma linha válida de código, pois a parte do comentário será ignorada quando você rodar. No SC, usamos um ponto-e-vírgula para indicar o fim de um enunciado válido.

\begin{lstlisting}[style=SuperCollider-IDE, basicstyle=\scttfamily\footnotesize]
2 + 5 + 10 - 5; // apenas fazendo contas

rrand(10, 20); // gerar um numero aleatório entre 10 e 20
\end{lstlisting}

Você pode rodar uma linha mesmo que o seu cursor estiver no meio de um comentário depois desta linha. A parte do comentário é ignorada. Os próximos dois parágrafos serão escritos como "comentários" apenas como exemplo.


 
\begin{lstlisting}[style=SuperCollider-IDE, basicstyle=\scttfamily\footnotesize]
// Você pode rapidamente transformar uma linha de código em comentário usando o atalho [ctrl+/].
"Algum código de SC aqui...".postln;
2 + 2;


// Se você escrever um comentário beeeem longo (uma única linha longa), seu texto vai ser quebrado em várias "linhas" que não vão começar com barra dupla. No entanto, tudo isso ainda conta como uma só linha de comentário.

/* Use "barra + asterisco" para começar um comentário longo com diversas linhas. Feche o trecho de comentário com "asterisco + barra". O atalho mencionado anteriormente também funciona para grandes trechos: simplesmente selecione as linhas de código que você quer "comentar" ("comment out", em inglês) e pressione [ctrl+/]. O mesmo serve para des-comentar ("uncomment").*/
\end{lstlisting}
