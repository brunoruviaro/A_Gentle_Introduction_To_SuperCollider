\section{Blocos de código}
\label{sec:code-block}


Selecionando múltiplas linhas de um código antes de rodá-lo pode ser tedioso. Uma maneira muito mais fácil de rodar toda uma porção de código ao mesmo tempo é criar um \textit{bloco de código}: simplesmente coloque dentro de parênteses todas as linhas de código que você quer rodar juntas. Aqui temos um exemplo:


\begin{lstlisting}[style=SuperCollider-IDE, basicstyle=\scttfamily\footnotesize]
(
// Um pequeno poema
"Hoje é domingo".postln;
"Pé de cachimbo".postln;
"O cachimbo é de ouro".postln;
"Bate no touro".postln;
)
\end{lstlisting}

Os parênteses externos delimitam o bloco de código. Desde que que o cursor esteja em qualquer lugar dentro dos parênteses, um único [ctrl+Enter] rodará as linhas para você (elas serão executadas na ordem de cima para baixo, mas isso é tão rápido que parece simultâneo).

Usar blocos de código poupa o trabalho de ter de selecionar todas as linhas novamente a cada vez que você quiser mudar algo e rodar novamente. Por exemplo mude algumas das palavras entre aspas e pressione [ctrl+Enter] logo após fazer a mudança. Todo o bloco de código é rodado sem que você tenha que selecionar manualmente todas as linhas. O SuperCollider ressalta o bloco por um segundo para dar uma indicação visual de o que está sendo executado.
