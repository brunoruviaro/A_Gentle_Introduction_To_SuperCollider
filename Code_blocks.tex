\section{Code blocks}
\label{sec:code-block}

Selecting multiple lines of code before evaluating can be tedious. A much easier way of running a chunk of code all at once is by creating a \textit{code block}: simply enclose in parentheses all lines of code that you want to run together. Here's an example:
 
\begin{lstlisting}[style=SuperCollider-IDE, basicstyle=\scttfamily\footnotesize]
(
// A little poem
"Today is Sunday".postln;
"Foot of pipe".postln;
"The pipe is made of gold".postln;
"It can beat the bull".postln;
)
\end{lstlisting}

The outer parentheses are delimiting the code block. As long as your cursor is anywhere within the parentheses, a single [ctrl+Enter] will evaluate all lines for you (they are executed in order from top to bottom, but it's so fast that it seems simultaneous).

Using code blocks saves you the trouble of having to select all the lines again every time you change something and want to re-evaluate. For example, change some of the words between double quotes, and hit [ctrl+Enter] right after making the change. The entire block of code is evaluated without you having to manually select all lines. SuperCollider highlights the block for a second to give you a visual hint of what's being executed.