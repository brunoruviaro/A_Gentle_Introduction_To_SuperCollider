\section{MIDI}

Uma apresentação aprofundada dos conceitos e truques do MIDI está além do escopo deste tutorial. Os exemplos abaixo assumem alguma familiaridade com dispositivos MIDI e são fornecidos apenas como uma introdução.


\begin{lstlisting}[style=SuperCollider-IDE, basicstyle=\scttfamily\footnotesize]
// Jeito rápido de conectar todos os dispositivos disponíveis ao SC
MIDIIn.connectAll;

// Jeito rápido de ver todas as mensagens MIDI que estão chegando
MIDIFunc.trace(true);
MIDIFunc.trace(false); // pare com isso

// Jeito rápido de inspecionar todas as entradas CC
MIDIdef.cc(\someCC, {arg a, b; [a, b].postln});

// Obter entrada somente do cc 7, canal 0
MIDIdef.cc(\algumControleEspecifico, {arg a, b; [a, b].postln}, ccNum: 7, chan: 0);

// Um SynthDef para testes rápidos
SynthDef(“rápido”, {arg freq, amp; Out.ar(0, SinOsc.ar(freq) * Env.perc(level: amp).kr(2))}).add;

// Toque de um teclado ou pad de percussão
(
MIDIdef.noteOn(\algumTeclado, { arg vel, nota;
	Synth("rápido", [\freq, nota.midicps, \amp, vel.linlin(0, 127, 0, 1)]);
});
)

// Criar um padrão e tocá-lo com um teclado
(
a = Pbind(
	\instrument, “rápido”,
	\degree, Pwhite(0, 10, 5),
	\amp, Pwhite(0.05, 0.2),
	\dur, 0.1
);
)

// teste
a.play;

// Disparar padrão de um pad ou teclado
MIDIdef.noteOn(\quneo, {arg vel, note; a.play});
\end{lstlisting}

Uma dúvida frequente é como administrar mensagens de “liga nota” e “desliga nota” (“note on” e “note off”) para notas sustentadas. Em outras palavras, quando você utiliza um envelope ADSR, você quer que cada nota seja sustentada enquanto a tecla estiver pressionada. O estágio de repouso (“release”) inicia apenas quando o dedo sai da tecla correspondente (revise a seção sobre envelopes ADSR se necessário).

Para fazer isso, o SuperCollider simplesmente precisa monitorar qual nó de sintetizador corresponde a cada tecla. Podemos usar um Array para este fim, como demonstrado no exemplo abaixo. 

\begin{lstlisting}[style=SuperCollider-IDE, basicstyle=\scttfamily\footnotesize]
 // Um SynthDef com envelope ADSR
SynthDef(“rápido2", {arg freq = 440, amp = 0.1, gate = 1;
	var snd, env;
	env = Env.adsr(0.01, 0.1, 0.3, 2, amp).kr(2, gate);
	snd = Saw.ar([freq, freq*1.5], env);	
	Out.ar(0, snd)
}).add;

// Toque com um teclado MIDI

(
var arrayDeNotas = Array.newClear(128); // array com uma posição para cada nota MIDI possível

MIDIdef.noteOn(\minhaTeclaApertada, {arg vel, nota;
	arrayDeNotas[nota] = Synth(“rápido2", [\freq, nota.midicps, \amp, vel.linlin(0, 127, 0, 1)]);
	["NOTA LIGADA", nota].postln;
});
	
MIDIdef.noteOff(\minhaTeclaLiberada, {arg vel, nota;
	arrayDeNotas[nota].set(\gate, 0);
	["NOTA DESLIGADA”, nota].postln;
});
)
// PS. Garanta que as conexões MIDI do SC estejam ativas (MIDIIn.connectAll)
 \end{lstlisting} 
 
Para ajudar a entender o código acima:

\begin{itemize}
\item O SynthDef \texttt{“rápido2”} usa um envelope ADSR. O argumento \texttt{gate} é responsável por ligar e desligar as notas.
\item Um Array chamado \texttt{“arrayDeNotas”} é criado para monitorar quais notas estão sendo tocadas. Os índices do array devem corresponder aos números das notas MIDI sendo tocadas.
\item Toda vez que uma tecla é pressionada no teclado, um Synth começa a tocar (um nó de sintetizador é criado no servidor) e \emph{a referência a este nó de sintetizador é armazenada em uma posição exclusiva no array}; o índice do array é simplesmente o próprio número de nota MIDI.
\item Sempre que a tecla é liberada, a mensagem \texttt{.set(\textbackslash gate, 0)} é enviada para o nó de sintetizador apropriado, recuperado do array através do número da nota.
\end{itemize}

Nesta curta demonstração de MIDI apenas discutimos como enviar informação MIDI \emph{para dentro} do SuperCollider. Para enviar mensagens MIDI \emph{para fora} do SuperCollider, dê uma olhada no arquivo de Ajuda de \texttt{MIDIOut}.
