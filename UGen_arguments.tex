\section{Argumentos de UGen}

Quase sempre você vai precisar especificar os argumentos das UGens que você estiver usando. Você já viu isso antes: quando escrevemos \texttt{\{SinOsc.ar(440)\}.play}, o número \texttt{440} é um argumento para o \texttt{SinOsc.ar}; ele especifica a frequência que você vai ouvir. Você pode nomear os argumentos explicitamente, assim: \texttt{\{SinOsc.ar(freq: 440, mul: 0.5)\}.play}. Os nomes dos argumentos são \texttt{freq} e \texttt{mul} (note os dois pontos imediatamente após as palavras no código). O \texttt{mul} quer dizer "multiplicador" e é essencialmente a amplitude da forma de onda. Se você não prover nenhum valor para \texttt{mul}, o SuperCollider usa o valor padrão de 1 (amplitude máxima). Um valor como \texttt{mul: 0.5} significa multiplicar a forma de onda por meio, em outras palavras, ela vai tocar com metade da amplitude máxima.

No código do seu theremin, os argumentos \texttt{freq} e \texttt{mul} do \texttt{SinOsc} foram explicitamente nomeados. Você deve lembrar que \texttt{MouseX.kr(300, 2500)} foi usado para controlar a frequência do theremin. Nesse caso, o \texttt{MouseX.kr} recebeu dois argumentos: um limite mínimo e um máximo para seu âmbito de saída (300 e 2500, respectivamente). O mesmo vale para o \texttt{MouseY.kr(0, 1)}, controlando a amplitude. Esses argumentos dentro as UGens de mouse não foram explicitamente nomeados, mas poderiam ter sido.

Como fazer pra descobrir que argumentos uma UGen aceita? Simplesmente vá para o arquivo de Ajuda correspondente: dê um duplo clique no nome da UGen para selecioná-la e aperte [ctrl+D] para abrir a página de documentação. Faça isso, digamos, com o MouseX. Depois da seção Description ("Descrição") você verá a seção Class Methods ("Métodos da Classe"). É ali que você vai descobrir que os argumentos do método \texttt{kr} são minval, maxval, warp e lag. Na mesma página, você também encontra a explicação sobre o que cada um deles significa.

Sempre que você não fornece um argumento, o SC vai usar os valores padrão que você vê no arquivo de Ajuda. Se você não nomear explicitamente os argumentos, você terá que fornecê-los na ordem exata mostrada no arquivo de Ajuda. Se você nomeá-los explicitamente, você pode colocá-los em qualquer ordem e até mesmo pular alguns do meio. Nomear explicitamente os argumentos é também uma boa ferramente de aprendizado, pois te ajuda a entender melhor o código. Um exemplo é dado abaixo.

\begin{lstlisting}[style=SuperCollider-IDE, basicstyle=\scttfamily\footnotesize]
// minval e maxval fornecidos na ordem, sem palavras-chave
{MouseX.kr(300, 2500).poll}.play;
// minval, maxval e lag fornecidos, warp foi pulado
{MouseX.kr(minval: 300, maxval: 2500, lag: 10).poll}.play;
\end{lstlisting}
