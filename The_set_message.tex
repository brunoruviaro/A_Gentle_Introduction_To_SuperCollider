\section{A mensagem \texttt{set}}

Assim como com qualquer função (reveja a seção \ref{sec:functions}), argumentos especificados no início da sua função de sintetizador ficam acessíveis ao usuário. Isso permite que você modifique parâmetros em tempo real (enquanto o sintetizador está rodando). A mensagem \texttt{set} ("definir") é usada para este fim. Exemplo simples:

\begin{lstlisting}[style=SuperCollider-IDE, basicstyle=\scttfamily\footnotesize]
x = {arg freq = 440, amp = 0.1; SinOsc.ar(freq, 0, amp)}.play;
x.set(\freq, 778);
x.set(\amp, 0.5);
x.set(\freq, 920, \amp, 0.2);
x.free;
\end{lstlisting}

É um bom hábito fornecer valores pré-definidos (como 440 e 0.1 acima), de outra forma o sintetizador não vai tocar até que você defina um valor apropriado para os parâmetros 'vazios'.
