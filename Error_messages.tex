\section{Mensagens de erro}

Não saiu nenhum som quando você rodou o último exemplo? Se isso aconteceu, provavelemente seu código tem um erro de digitação: um caracter errado, uma vírgula ou um parêntese a menos, etc. Quando algo acontece de errado no seu código, a Post window mostra uma mensagem de erro. Mensagens de erro podem ser longas e obscuras, mas não entre em pânico: com o tempo você aprenderá a lê-las. Veja abaixo um exemplo de uma mensagem de erro curta:

\begin{verbatim}
ERROR: Class not defined.
  in file 'selected text'
  line 1 char 19:

  {SinOsc.ar(LFNoiseO.kr(12).range(400, 1600), mul: 0.01)}.play; 
                     
-----------------------------------
nil
\end{verbatim}

Esta mensagem de erro diz "Class not defined" (Classe não definida) e aponta a localização aproximada do erro ("line 1 char 19", ou seja: linha 1, caracter 19). Classes no SC são aquelas palavras azuis que começam com uma letra maiúscula (como \texttt{SinOsc} e \texttt{LFNoise0}). O que causou o erro nesse exemplo foi que a pessoa digitou LFNoiseO com uma letra "O" maiúscula ao final. A classe correta é LFNoise0, com o número zero ao final. Parece até pegadinha de vestibular, mas é verdade. Como você pode ver, atenção aos detalhes é crucial.

Se você tem um erro no seu código, revise-o, mude o que for necessário e tente novamente até que ele esteja corrigido. Se você ainda não cometeu nenhum erro, experimente introduzir um para que você possa ver como é uma mensagem de erro (por exemplo, remova um ponto ou uma vírgula de um dos exemplos das senoides).

\bigskip
\todo[inline, color=green!40]{ 
DICA: Aprender SuperCollider é como aprender uma outra língua como Alemão, Inglês ou Japonês\dots\  você tem que praticar falá-la o máximo possível, esforce-se em expandir seu vocabulário, preste atenção na gramática e na sintaxe e aprenda com seus erros. A pior coisa que pode acontecer é você travar o SuperCollider, o que é bem menos ruim do que pegar um ônibus errado em Nova Iorque por culpa de um erro de pronúncia na hora de pedir informações.
}
