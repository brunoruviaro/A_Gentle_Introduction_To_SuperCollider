\section{Aninhamento ("Nesting"}
\label{sec:nesting}

A solução do último exercício levou você a aninhar coisas uma dentro da outra, isso é, você colocou o rand dentro do round que por sua vez foi pra dentro do dup. David Cottle tem uma explicação excelente para aninhamento no The SuperCollider Book, então simplesmente o citaremos aqui.\footnote{Cottle, D. ``Beginner's Tutorial.'' The SuperCollider Book, MIT Press, 2011, pp. 8-9.}

\begin{quotation}
\textit{Para explicar melhor a ideia de aninhamento, considere um exemplo hipotético no qual o SC vai preparar sua refeição. Para fazer isso você pode usar uma mensagem} \texttt{servir}. \textit{Os argumentos podem ser salada, prato principal e sobremesa. Mas apenas dizer} \texttt{servir(alface, peixe, banana)} \textit{talvez não produza o resultado que você quer. Pra ter certeza que a comida seja bem feita, você pode explicar melhor os argumentos, substituindo-os por uma mensagem aninhada e alguns argumentos mais específicos.}
\end{quotation}

\texttt{servir(misturar(alface, tomate, queijo), assar(peixe, 400, 20), bater(banana, sorvete))
}
\begin{quotation}
\textit{o SC então não apenas servirá alface, peixe e banana, mas uma salada mista com alface, tomate e queijo; um peixe assado; e um sundae de banana. Estes comandos internos podem ser ainda mais explicados, aninhando uma mensagem(arg) para cada ingrediente: alface, tomate, queijo e assim por diante. Cada mensagem interna produz um resultado que por sua vez é usado como argumento pela mensagem exterior.}
\end{quotation}

\begin{lstlisting}[style=SuperCollider-IDE, basicstyle=\scttfamily\footnotesize, label=code-dinner]
// Pseudo-código para fazer o jantar: 
servir(
	misturar(
		lavar(alface, água, 10),
		picar(tomate, pequeno),
		salpicar(escolher([gorgonzola, feta, gouda]))
	),
	assar(pescar(lagoa, anzol, vara), 200, 20),
	misturar(
		fatiar(descascar(banana), 20),
		cozinhar(misturar(leite, açúcar, amido), 200, 10)
	)
);
\end{lstlisting}

\begin{quotation}
\textit{Quando o aninhamento tem diversos níveis, podemos usar novas linhas e indentações para uma maior clareza. Algumas mensagens e argumentos podem permanecer na mesma linha, enquanto outras mensagens e argumentos podem distribuídos um em cada linha---o que quer seja mais claro. Cada nível de indentação deve indicar um nível de aninhamento. (Note que você pode ter qualquer quantidade de espaço em branco---novas linhas, tabulações e espaços---entre trechos de código.)}

[No exemplo do jantar,]\textit{ agora pede-se ao programa de refeições que ele lave a alface em água por 10 minutos e pique o tomate em pequenos pedaços antes de misturá-los na travessa da salada e salpicá-los com queijo. Você também especificou onde pegar o peixe e pediu para assá-lo a 200 graus por 20 minutos antes de servir, e assim por diante.} 
\textit{Para "ler" este estilo de código, você começa da mensagem aninhada mais interna e vai seguindo para fora camada por camada. Aqui está um exemplo alinhado de maneira a mostrar como a mensagem mais interna é aninhada dentro das outras mensagens.}
\end{quotation}

%\lstinputlisting[style=SuperCollider-IDE, basicstyle=\scttfamily\footnotesize]{code-nesting.scd}

\begin{lstlisting}[style=SuperCollider-IDE, basicstyle=\scttfamily\footnotesize]
                exprand(1.0, 1000.0);
           dup({exprand(1.0, 1000.0)}, 100);
      sort(dup({exprand(1.0, 1000.0)}, 100));
round(sort(dup({exprand(1.0, 1000.0)}, 100)), 0.01);
\end{lstlisting}

O código abaixo é um outro exemplo de aninhamento. Responda às perguntas que se seguem. Você não precisa explicar o que os números estão fazendo---a tarefa é simplesmente idenftificar os argumentos em cada camada de aninhamento. (Este exemplo e as questões do exercício também são emprestadas e ligeiramente modificadas do tutorial do Cottle.)

 
%\lstinputlisting[style=SuperCollider-IDE, basicstyle=\scttfamily\footnotesize, label=code-nested-music]{code-nested-music.scd}

\begin{lstlisting}[style=SuperCollider-IDE, basicstyle=\scttfamily\footnotesize]
// Aninhamento e indentação apropriada
(
{
	CombN.ar(
		SinOsc.ar(
			midicps(
				LFNoise1.ar(3, 24,
					LFSaw.ar([5, 5.123], 0, 3, 80)
				)
			),
			0, 0.4
		),
		1, 0.3, 2)
}.play;
)
\end{lstlisting}

\begin{enumerate}[a)]
\item Qual número é o segundo argumento do \texttt{LFNoise1.ar}?
\item Qual o primeiro argumento do \texttt{LFSaw.ar}?
\item Qual o terceiro argumento do \texttt{LFNoise1.ar}?
\item O método \texttt{midicps} tem quantos argumentos?
\item Qual o terceiro argumento do \texttt{SinOsc.ar}?
\item Qual o segundo e terceiro argumentos do \texttt{CombN.ar}?
\end{enumerate}

Confira as respostas ao final do livro.\endnote{ Respostas: 
\begin{enumerate}[a)]
\item 24
\item $\left[5, 5.123\right]$ (tanto números quanto colchetes)
\item Toda a linha do \texttt{LFSaw}
\item Somente um
\item 0.4
\item 1 e 0.3
\end{enumerate}
 }

\medskip
 
\bigskip
\todo[inline, color=green!40]{ 
DICA: Se por qualquer motivo, seu código perdeu a indentação apropriada, simplesmente selecione tudo e vá para o menu Edit$\rightarrow$Autoindent Line or Region ("Autoindentar Linha ou Região") e isto será consertado.
}
\bigskip
