\section{More Pattern tricks}

\subsection{Chords}

Want to write chords inside \texttt{Pbind}s? Write them as lists (comma-separated values enclosed in square brackets):
 
\begin{lstlisting}[style=SuperCollider-IDE, basicstyle=\scttfamily\footnotesize]
(
Pbind(
	\note, Pseq([[0, 3, 7], [2, 5, 8], [3, 7, 10], [5, 8, 12]], 3),
	\dur, 0.15
).play;
)
// Fun with strum
(
Pbind(
	\note, Pseq([[-7, 3, 7, 10], [0, 3, 5, 8]], 2),
	\dur, 1,
	\legato, 0.4,
	\strum, 0.1 // try 0, 0.1, 0.2, etc
).play;
)
\end{lstlisting}
 

\subsection{Scales}

When using \texttt{\textbackslash degree} for your pitch specification, you can add another line with the keyword \texttt{\textbackslash scale} to change scales (note: this only works in conjunction with \texttt{\textbackslash degree}, not with \texttt{\textbackslash note}, \texttt{\textbackslash midinote}, or \texttt{\textbackslash freq}):

 
\begin{lstlisting}[style=SuperCollider-IDE, basicstyle=\scttfamily\footnotesize]
(
Pbind(
	\scale, Scale.harmonicMinor,
	\degree, Pseq([0, 1, 2, 3, 4, 5, 6, 7], 1),
	\dur, 0.15;
).play;
)

// Evaluate this line to see a list of all available scales:
Scale.directory;

// If you need a chromatic note in between scale degrees, do this:
(
Pbind(
	\degree, Pseq([0, 1, 2, 3, 3.1, 4], 1),
).play;
)

// The 3.1 above means one chromatic step above scale degree 3
// (in this case, F# above F). Note that when you don't explicitly
// specify a \scale, Scale.major is assumed.
\end{lstlisting}

\subsection{Transposition}

Use the \texttt{\textbackslash ctranspose} keyword to achieve chromatic transposition. This will work in conjunction with \texttt{\textbackslash degree}, \texttt{\textbackslash note}, and \texttt{\textbackslash midinote}, but not with \texttt{\textbackslash freq}.

\begin{lstlisting}[style=SuperCollider-IDE, basicstyle=\scttfamily\footnotesize]
(
Pbind(
	\note, Pser([0, 2, 3, 5, 7, 8, 11, 12], 11),
	\ctranspose, 12, // transpose an octave above (= 12 semitones)
	\dur, 0.15;
).play;
)
\end{lstlisting}

\subsection{Microtones}
 
How to write microtones:

\begin{lstlisting}[style=SuperCollider-IDE, basicstyle=\scttfamily\footnotesize]
// Microtones with \note and \midinote:
Pbind(\note, Pseq([0, 0.5, 1, 1.5, 1.75, 2], 1)).play;
Pbind(\midinote, Pseq([60, 69, 68.5, 60.25, 70], 1)).play;
\end{lstlisting}
 
\subsection{Tempo}

The values you provide to the \texttt{\textbackslash dur} key of a Pbind are in \emph{number of beats}, that is, 1 means one beat, 0.5 means half a beat, and so on. Unless you specify otherwise, the default tempo is 60 BPM (beats per minute). To play at a different tempo, you simply create a new TempoClock. Here's a \texttt{Pbind} playing at 120 beats per minute (BPM):
 
\begin{lstlisting}[style=SuperCollider-IDE, basicstyle=\scttfamily\footnotesize]
(
Pbind(\degree, Pseq([0, 0.1, 1, 2, 3, 4, 5, 6, 7]),
	\dur, 1;
).play(TempoClock(120/60)); // 120 beats over 60 seconds: 120 BPM
)
\end{lstlisting}
 

By the way, did you see that the \texttt{Pseq} above is taking only one argument (the list)? Where is the \texttt{repeats} value that always came after the list? You can hear that the example plays through the sequence only once, but why? This is a common property of all Patterns (and in fact, of many other objects in SuperCollider): if you omit an argument, it will use a built-in default value. In this case, the default \texttt{repeats} for \texttt{Pseq} is 1. Remember your first ridiculously simple \texttt{Pbind}? It was a mere \texttt{Pbind(\textbackslash degree, 0).play} and it only knew how to play one note. You didn't provide any info for duration, amplitude, legato, etc. In theses cases \texttt{Pbind} simply goes ahead and uses its default values for those.

\subsection{Rests}

Here is how to write rests. The number inside \texttt{Rest} is the duration of the rest in beats. Rests can go anywhere in the Pbind, not just in the \texttt{\textbackslash dur} line.

 
\begin{lstlisting}[style=SuperCollider-IDE, basicstyle=\scttfamily\footnotesize]
(
Pbind(
	\degree, Pwhite(0, 10),
	\dur, Pseq([0.1, 0.1, 0.3, 0.6, Rest(0.3), 0.25], inf);
).play;
)
\end{lstlisting}
 

\subsection{Playing two or more Pbinds together}

To start a few Pbinds simultaneously, simply enclose all of them within a single code block:
 
\begin{lstlisting}[style=SuperCollider-IDE, basicstyle=\scttfamily\footnotesize]
( // open big block
Pbind(
	\freq, Pn(Pseries(110, 111, 10)),
	\dur, 1/2,
	\legato, Pwhite(0.1, 1)
).play;

Pbind(
	\freq, Pn(Pseries(220, 222, 10)),
	\dur, 1/4,
	\legato, Pwhite(0.1, 1)
).play;

Pbind(
	\freq, Pn(Pseries(330, 333, 10)),
	\dur, 1/6,
	\legato, 0.1
).play;
) // close big block
\end{lstlisting}

In order to play Pbinds in a time-ordered fashion (other than simply evaluating them manually one after the other), you can use \texttt{\{ \}.fork}:

\begin{lstlisting}[style=SuperCollider-IDE, basicstyle=\scttfamily\footnotesize]
// Basic fork example. Watch Post window:
( 
{
	"one thing".postln;
	2.wait;
	"another thing".postln;
	1.5.wait;
	"one last thing".postln;
}.fork;
)
// A more interesting example:
(
t = TempoClock(76/60);
{
	Pbind(
		\note, Pseq([[4, 11], [6, 9]], 32),
		\dur, 1/6,
		\amp, Pseq([0.05, 0.03], inf)
	).play(t);
	
	2.wait;
	
	Pbind(
		\note, Pseq([[-25, -13, -1], [-20, -8, 4], \rest], 3),
		\dur, Pseq([1, 1, Rest(1)], inf),
		\amp, 0.1,
		\legato, Pseq([0.4, 0.7, \rest], inf)
	).play(t);

	2.75.wait;
	
	Pbind(
		\note, Pseq([23, 21, 25, 23, 21, 20, 18, 16, 20, 21, 23, 21], inf),
		\dur, Pseq([0.25, 0.75, 0.25, 1.75, 0.125, 0.125, 0.80, 0.20, 0.125, 0.125, 1], 1),
		\amp, 0.1,
		\legato, 0.5
	).play(t);
}.fork(t);
)
\end{lstlisting}
 
For advanced ways of playing \texttt{Pbind}s simultaneously and in sequence, check out \texttt{Ppar} and \texttt{Pspawner}. For more about \texttt{fork}, check out the \texttt{Routine} Help file.

\subsection{Using variables}

In the earlier section ``Expanding your Pattern vocabulary,'' did you notice how you had to type the same note list [0, 2, 3, 5, 7, 8, 11, 12] several times for multiple \texttt{Pbind}s? Not very efficient to copy the same thing by hand over and over, right? In programming, whenever you find yourself doing the same task repeatedly, it's probably time to adopt a smarter strategy to accomplish the same goal. In this case, we can use variables. As you may remember, variables allow you to refer to any chunk of data in a flexible and concise way (review section  \ref{sec:variables} if needed). Here's an example:

\begin{lstlisting}[style=SuperCollider-IDE, basicstyle=\scttfamily\footnotesize]
// Using the same sequence of numbers a lot? Save it in a variable:
c = [0, 2, 3, 5, 7, 8, 11, 12];

// Now you can just refer to it:
Pbind(\note, Pseq(c, 1), \dur, 0.15).play;
Pbind(\note, Prand(c, 6), \dur, 0.15).play;
Pbind(\note, Pslide(c, 5, 3, 1), \dur, 0.15).play;
\end{lstlisting}
 
Another example to practice using variables: let's say we want to play two \texttt{Pbind}s simultaneously. One of them does an ascending major scale, the other does a descending major scale one octave above. Both use the same list of durations. Here is one way of writing this:
 
\begin{lstlisting}[style=SuperCollider-IDE, basicstyle=\scttfamily\footnotesize]
~scale = [0, 1, 2, 3, 4, 5, 6, 7];
~durs = [0.4, 0.2, 0.2, 0.4, 0.8, 0.2, 0.2, 0.2];
(
Pbind(
	\degree, Pseq(~scale),
	\dur, Pseq(~durs)
).play;

Pbind(
	\degree, Pseq(~scale.reverse + 7),
	\dur, Pseq(~durs)
).play;
)
\end{lstlisting}
 
Interesting tricks here: thanks to variables, we reuse the same list of scale degrees and durations for both \texttt{Pbind}s. We wanted the second scale to be descending and one octave above the first. To achieve this, we simply use the message \texttt{.reverse} to reverse the order of the list (type \texttt{$\sim$scale.reverse} on a new line and evaluate to see exactly what it does). Then we add 7 to transpose it one octave above (test it as well to see the result).\footnote{We could also have used \texttt{\textbackslash ctranspose, 12} to get the same transposition.} We played two \texttt{Pbind}s at the same time by enclosing them within a single code block.

Exercise: create one additional \texttt{Pbind} inside the code block above, so that you hear three simultaneous voices. Use both variables (\texttt{$\sim$scale} and \texttt{$\sim$durs}) in some different way---for example, use them inside a pattern other than Pseq; change transposition amount; reverse and/or multiply durations; etc.

\subsection{Assignment to multiple keys}

This alternative way of writing allows you to assign multiple keys in the form of Arrays. In the example below, one might find useful to write in this way in order to keep notes and durations together as pairs:

\begin{lstlisting}[style=SuperCollider-IDE, basicstyle=\scttfamily\footnotesize]
// assignment to multiple keys
(
Pbind(
	#[midinote, dur], Pseq([
		[60, 0.75],
		[67, 0.25]
	], inf),
	\amp, 0.7
).play;
)
\end{lstlisting}