\section{Notação de objeto recebedor, notação funcional}


Há mais de uma maneira de escrever expressões no SuperCollider. A que vimos acima é chamada \emph{notação de objeto recebedor} (“receiver notation”): \texttt{100.rand}, na qual um ponto conecta o Objeto (\texttt{100}) à mensagem (\textt(rand}. Alternativamente, exatamente a mesma coisa pode ser escrita assim: \texttt{rand(100)}. Isso é chamado \emph{notação funcional} ("funcional notation"). 

Você pode escrever das duas formas. Eis como isso funciona quando uma mensagem recebe dois ou mais argumentos.

 
\begin{lstlisting}[style=SuperCollider-IDE, basicstyle=\scttfamily\footnotesize]
5.dup(20);  // notação de objeto recebedor
dup(5, 20); // mesma coisa em notação funcional

3.1415.round(0.1); // notação de objeto recebedor
round(3.1415, 0.1); // notação funcional
\end{lstlisting}
 

Nos exemplos acima, você pode ler \texttt{dup(5, 20)} como “faça duplicatas do número 5 vinte vezes” e \texttt{round(3.1415, 0.1)} como “arredonde o número 3.1415 para uma casa decimal”. Por sua vez, as versões em notação de objeto recebedor podem ser lidas como “Número 5, faça duplicatas de si mesmo vinte vezes!” (para \texttt{5.dup(20)}) e “Número 3.1415, arredonde-se para uma casa decimal!'' (para \texttt{3.1415.round(0.1)}).

Resumindo: \texttt{Recebedor.mensagem(argumento)} é equivalente a \texttt{mensagem(Recebedor, argumento)}.

Escolher um estilo de escrita em vez do outro é uma questão de preferência pessoal e convenção. Às vezes um método pode ser mais claro que o outro. Qualquer que seja o estilo que você acabe preferindo (e tudo bem misturá-los), o importante é ser consistente. Uma convenção que é muito difundida entre usuários de SuperCollider é que classer (palavras que começam com letras maiúsculas) são quase sempre escritas como \texttt{Recebedor.mensagem(argumento)}. Por exemplo, você sempre verá \texttt{SinOsc.ar(440)}, mas quase nunca verá \texttt{ar(SinOsc, 440)}, embora ambas estejam corretas.

Exercício: reescreva a seguinte sentença usando apenas notação funcional:

\texttt{100.0.rand.round(0.01).dup(4);} 

Solução ao final.\endnote{Reescrevendo usando apenas notação funcional: \texttt{dup(round(rand(100.0), 0.01), 4);}}
