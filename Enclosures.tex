\section{Fechamentos}

Há quatro tipos de fechamento: \texttt{(parênteses)}, \texttt{[colchetes]}, \texttt{\{chaves\}} e \texttt{"aspas duplas"}.

Cada um que você abre, deve ser fechado mais adiante. Isso é chamado "balancear", ou seja, manter devida correspondência entre os pares de fechamento em todo o seu código.

A IDE do SuperCollider automaticamente indica o fechamento de parênteses (também de colchetes e chaves) quando você fecha um par--- eles são destacados em vermelho. Se você clicar um parênteses que não tem um par de abertura/fechamento, você verá uma seleção em vermelho escuro indicando que algo está faltando. Balancear é uma maneira rápida de selecionar grandes seções de código para ser executado, deletado ou para operações de copiar/colar. Você pode fazer um clique duplo em um parênteses de abertura ou fechamento para selecionar tudo o que está contido (o mesmo vale para colchetes e chaves).

\subsection{Aspas duplas}

Aspas duplas são usadas para englobar uma sequência de caracteres (incluindo espaços) como uma única unidade. Estas são chamadas Strings (“cadeias”). Aspas simples criam Símbolos ("Symbols"), que são ligeiramente diferentes de Strings. Símbolos também podem ser criados com uma barra invertida imediatamente antes do texto. Portanto, \texttt{'grandeSimbolo'} and \texttt{\textbackslash grandeSimbolo} são equivalentes.

\begin{lstlisting}[style=SuperCollider-IDE, basicstyle=\scttfamily\footnotesize]
"Aqui está um bom string";
'grandeSimbolo';
\end{lstlisting}

\subsection{Parênteses}

Parênteses podem ser usados para:

\begin{itemize}
\item englobar listas de argumentos: \texttt{rrand(0, 10);}
\item forçar precedência: \texttt{5 + (10 * 4);}
\item criar blocos de código (múltiplas linhas de código para serem rodadas simultaneamente).
\end{itemize} 

\subsection{Colchetes}

Colchetes definem uma coleção de itens, como \texttt{[1, 2, 3, 4, "hello"]}. Estas são normalmente chamadas Arrays. Uma array pode conter qualquer coisa: números, strings, funções, padrões, etc. Arrays entendem mensagens como \texttt{reverse} (“inverter”), \texttt{scramble} (“embaralhar”), \texttt{mirror} (“espelhar”), \texttt{choose} (“escolher”), para mencionar algumas. Você também pode fazer operações matemáticas em arrays.


\begin{lstlisting}[style=SuperCollider-IDE, basicstyle=\scttfamily\footnotesize]
[1, 2, 3, 4, “alô”].scramble;
[1, 2, 3, 4, "alô"].mirror;
[1, 2, 3, 4].reverse + 10;
// converter notas MIDI para frequências em Hz 
[60, 62, 64, 65, 67, 69, 71].midicps.round(0.1);
\end{lstlisting}

Mais sobre arrays, em breve na seção \ref{sec:arrays}.

\subsection{Chaves}

Chaves definem funções. Funções encapsulam algum tipo de operação ou tarefa que será provavelmente reutilizada múltiplas vezes, possivelmente retornando diferentes resultados a cada vez. O exemplo abaixo é do The SuperCollider Book:

\begin{lstlisting}[style=SuperCollider-IDE, basicstyle=\scttfamily\footnotesize]
exprand(1, 1000.0);
{exprand(1, 1000.0)};
\end{lstlisting}

David Cottle nos explica passo a passo seu exemplo: \textit{“a primeira linha seleciona um número randômico, que é mostrado na Post Window. O segundo imprime um resultado bastante diferente: uma função. O que a função faz? Ela seleciona um número randômico. Como pode esta diferença afetar o código? Considere as linhas abaixo. A primeira escolhe um número randômico e o duplica. A segunda executa  cinco vezes a função-seletora-de-números-randômicos e coleta os resultados em uma array.”}\footnote{Cottle, D. ``Beginner's Tutorial.'' The SuperCollider Book, MIT Press, 2011, p. 13.}

\begin{lstlisting}[style=SuperCollider-IDE, basicstyle=\scttfamily\footnotesize]
rand(1000.0).dup(5);  // seleciona um número e o duplica
{rand(1000.0)}.dup(5);  // duplica a função de selecionar um número {rand(1000.0)}.dup(5).round(0.1); // tudo o que foi feito acima, depois arredondando os valores
// essencialmente isto (que tem um resultado semelhante)
[rand(1000.0), rand(1000.0), rand(1000.0), rand(1000.0), rand(1000.0)]
\end{lstlisting}
 
Mais sobre funções em breve. Por ora, aqui está um resumo de todos os fechamentos possíveis:

\begin{description}
\item[Coleções] \texttt{[lista, de, itens]}
\item[Funções] \texttt{\{ quase sempre, múltiplas linhas de código \}}
\item[Strings] \texttt{“palavras entre aspas”}
\item[Símbolos] \texttt{‘aspasSimples’} ou precedidas de uma \texttt{\textbackslash barra invertida}
\end{description}
