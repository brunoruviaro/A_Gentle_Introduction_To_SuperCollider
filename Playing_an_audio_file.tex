\section{Tocando um arquivo de áudio}

Primeiro, você tem que carregar o arquivo de som em um buffer ("retentor"). O segundo argumento para \texttt{Buffer.read} é o caminho ("path") para o seu arquivo de som entre aspas duplas. Você precisará fazer com que ele aponte para um arquivo WAV ou AIFF no seu computador. Depois que os buffers são carregados, simplesmente use a UGen \texttt{PlayBuf} para tocá-lo de diversas maneiras.

\bigskip
\todo[inline, color=green!40]{ DICA: Um jeito rápido de obter o caminho correto para um arquivo de som no seu computador é arrastar o arquivo para um documento em branco do SuperCollider. O SC fornecerá automaticamente o camninho completo, já entre aspas duplas! }
\bigskip

\begin{lstlisting}[style=SuperCollider-IDE, basicstyle=\scttfamily\footnotesize]
// Carregar arquivos nos buffers:
~buf1 = Buffer.read(s, "/home/Music/wheels-mono.wav"); // um arquivo de som
~buf2 = Buffer.read(s, "/home/Music/mussorgsky.wav"); // outro arquivo de som

// Tocar:
{PlayBuf.ar(1, ~buf1)}.play; // número de canais e buffer
{PlayBuf.ar(1, ~buf2)}.play;

// Obter alguma informação sobre os arquivos:
[~buf1.bufnum, ~buf1.numChannels, ~buf1.path, ~buf1.numFrames];
[~buf2.bufnum, ~buf2.numChannels, ~buf2.path, ~buf2.numFrames];

// Mudar a velocidade de reprodução com 'rate' 
{PlayBuf.ar(numChannels: 1, bufnum: ~buf1, rate: 2, loop: 1)}.play;
{PlayBuf.ar(1, ~buf1, 0.5, loop: 1)}.play; // tocar na metade da velocidade
{PlayBuf.ar(1, ~buf1, Line.kr(0.5, 2, 10), loop: 1)}.play; // acelerando
{PlayBuf.ar(1, ~buf1, MouseY.kr(0.5, 3), loop: 1)}.play; // controlando com mouse

// Inverter direção (ao contrário)
{PlayBuf.ar(1, ~buf2, -1, loop: 1)}.play; // inverter som
{PlayBuf.ar(1, ~buf2, -0.5, loop: 1)}.play; // tocar o som na metade da velocidade E ao contrário
\end{lstlisting}
