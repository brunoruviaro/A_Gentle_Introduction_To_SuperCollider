\section{A última coisa é sempre postada}

Um detalhe pequeno mas útil para se entender: o SuperCollider tem como padrão sempre postar na Post window o resultado de qualquer operação que tenha sido a \emph{última coisa a ser executada}. Isso explica porque o seu código "Olá Mundo" imprime duas vezes quando você o executa. Digite as próximas linhas em um novo documento, depois selecione tudo com [ctrl+A] e rode todas as linhas de uma vez:

\begin{lstlisting}[style=SuperCollider-IDE, basicstyle=\scttfamily\footnotesize]
"Primeira linha".postln;
"Segunda linha".postln;
(2 + 2).postln;
3 + 3;
"Fim".postln;
\end{lstlisting}

Todas as cinco linhas são executadas pelo SuperCollider. Você vê o resultado de \texttt{2 + 2} na Post window porque existia um pedido de \texttt{postln} explícito. O resultado de \texttt{3 + 3} foi calculado, mas não foi dada nenhuma instrução para postá-lo, então você não o vê na Post window. Depois do 3 + 3, é executado o comando da última linha (a palavra "Fim" é postada por conta da solicitação \texttt{postln}). Finalmente, como padrão, o resultado da última coisa a ser rodada é postado: neste caso, acabou sendo a palavra "Fim".
