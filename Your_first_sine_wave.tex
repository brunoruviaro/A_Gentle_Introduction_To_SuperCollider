\section{Sua primeira senóide \label{sec:first-sine}}


"Oi Mundo" é tradicionalmente o primeiro programa que as pessoas criam quando estão aprendendo uma nova linguagem de programação. Você já fez isso no SuperCollider.

Criar uma onda senoidal simples talvez seja o "Olá Mundo" das linguagens de computador para música. Então vamos pular direto para isso. Digite e avalie a seguinte linha de código. Cuidado---isso pode ser alto. Abaixe todo seu volume, rode a linha e aumente o volume devagar.
 
\begin{lstlisting}[style=SuperCollider-IDE, basicstyle=\scttfamily\footnotesize]
{SinOsc.ar}.play;
\end{lstlisting}
 

Trata-se de uma senoide bela, suave, contínua e talvez um pouco entediante. Você pode parar o som com [ctrl+.] (Isso é a tecla \emph{control} mais a tecla de \emph{ponto final}.) Memorize esta combinação de teclas, porque você a utilizará muito para interromper todo e qualquer som no SC.

\bigskip
\todo[inline, color=green!40]{ 
DICA: Em uma linha separada, digite e rode \texttt{s.volume.gui} se você quiser um slider gráfico para controlar o volume da saída do SuperCollider.
}
\bigskip

Agora vamos tornar esta senoide um pouco mais interessante. Digite isso:

 
\begin{lstlisting}[style=SuperCollider-IDE, basicstyle=\scttfamily\footnotesize ]
{SinOsc.ar(LFNoise0.kr(10).range(500, 1500), mul: 0.1)}.play;
\end{lstlisting}
 

Lembre-se, você só precisa deixar o cursor em qualquer lugar da linha e apertar [ctrl+Enter] para executar. Alternativamente, você pode também selecionar toda a linha antes de executá-la.

 
\bigskip
\todo[inline, color=green!40]{ 
DICA: Digitar os próprios exemplos é uma grande ferramenta de aprendizagem. Isso irá ajudar a criar confiança e familiaridade com a linguagem. Quando ler tutoriais em formato digital, você pode sentir a tentação de simplesmente copiar e colar pequenos pedaços de código dos exemplos. Isso está bem, mas você aprenderá mais se digitar tudo por conta própria---tente ao menos nos primeiros estágios da sua aprendizagem com o SC.
}
