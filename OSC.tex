\section{OSC}

OSC (Open Sound Control) is a great way to communicate any kind of message between different applications or different computers over a network. In many cases, it is a much more flexible alternative to MIDI messages. We don't have space to explain it in more detail here, but the example below should serve as a good starting point.

The goal of the demo is to send OSC messages from a smartphone to your computer, or computer to computer.

On the receiver computer, evaluate this simple snippet of code:

\bigskip
\begin{lstlisting}[style=SuperCollider-IDE, basicstyle=\scttfamily\footnotesize]
(
OSCdef(
	key: \whatever,
	func: {arg ...args; args.postln},
	path: '/stuff')
)
\end{lstlisting}

Note: hitting [ctrl+.] will interrupt the \texttt{OSCdef} and you won't receive any more messages.

\subsection{Sending OSC from another computer}

This assumes that both computers are running SuperCollider and connected to a network. Find the IP address of the receiver computer, and evaluate the following lines in the sender computer:

\begin{lstlisting}[style=SuperCollider-IDE, basicstyle=\scttfamily\footnotesize]
// Use this on the machine sending messages
~destination = NetAddr("127.0.0.1", 57120); // use correct IP address of destination computer

~destination.sendMsg("/stuff", "heelloooo");
\end{lstlisting}


\subsection{Sending OSC from a smartphone}

\begin{itemize}
\item Install any free OSC app on the phone (for example, gyrosc);
\item Enter the IP address of the receiver computer into the OSC app (as 'target');
\item Enter SuperCollider's receiving port into the OSC app (usually 57120);
\item Check the exact message path that the app uses to send OSC to, and change your OSCdef accordingly;
\item Make sure your phone is connected to the network
\end{itemize}

As long as your phone is sending messages to the proper path, you should see them arriving on the computer.