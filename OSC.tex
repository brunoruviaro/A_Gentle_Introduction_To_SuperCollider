\section{OSC}

OSC (Open Sound Control ou “Controle de Som Aberto”) é uma excelente maneira de comunicar qualquer tipo de mensagem entre diferentes programas ou diferentes computadores em uma rede. Em muitos caos, é uma alternativa muito mais flexível às mensagens MIDI. Não temos espaço para explicar isso em mais detalhes aqui, mas o exemplo abaixo deve servir como um bom ponto de partida.

O objetivo desta demonstração é mandar mensagens OSC de um smartphone para seu computador ou de um computador para outro computador.

No computador receptor, rode este fragmento simples de código:

\bigskip
\begin{lstlisting}[style=SuperCollider-IDE, basicstyle=\scttfamily\footnotesize]
(
OSCdef(
	key: \seiLa,
	func: {arg ...args; args.postln},
	path: ‘/coisas’)
)
\end{lstlisting}

Nota: pressionando [ctrl+.] interromperá o \texttt{OSCdef} e você não receberá mais mensagens.

\subsection{Mandando OSC para um outro computador}

Isso assume que ambos os computadores estejam rodando o SuperCollider e conectados a uma rede. Descubra o endereço IP do computador receptor e rode as seguintes linhas no computador emissor:

\begin{lstlisting}[style=SuperCollider-IDE, basicstyle=\scttfamily\footnotesize]
// Use isso na máquina que está mandando mensagens
~destino = NetAddr("127.0.0.1", 57120); // use o endereço IP correto para o computador de destino

~destino.sendMsg(“/coisas”, “aaloooo");
\end{lstlisting}

\subsection{Mandando OSC de um smartphone}

\begin{itemize}
\item Instale qualquer aplicativo grátis de OSC no telefone (por exemplo, gyrosc);
\item Entre o endereço IP do computador receptor no aplicativo OSC (como ‘target’);
\item Entre a porta de entrada do SuperCollider no app OSC (geralmente 57120);
\item Verifique o caminho de mensagem (“message path”) que o aplicativo usa para mandar OSC e mude o seu OSCdef de acordo;
\item Tenha certeza que seu telefone está conectado à rede
\end{itemize}

Desde que seu telefone esteja enviando mensagens para o caminho correto, você deve vê-las chegando no computador.
