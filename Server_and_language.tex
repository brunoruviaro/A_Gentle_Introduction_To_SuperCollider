\section{Servidor e Linguagem}

Na Barra de Status você pode ver as palavras “Interpreter” (“Interpretador”) e “Server” (“Servidor”). Por definição, o Interpretador fica ligado (“Active”) na abertura do programa, enquanto o “Servidor” fica desligado (é isso o que todos aqueles zeros querem dizer). O que é o Interpretador é o que é o Servidor?

O SuperCollider, na realidade, é feito de dois aplicativos distintos: o servidor e a linguagem. O servidor é responsável por fazer sons. A linguagem (também chamada \emph{cliente} ou \emph{interpretador}) é usada para controlar o servidor. O primeiro é chamado scsynth (SC-synthesizer) e o segundo, sclang (SC-language). A Barra de Status diz o status (ligado/desligado) de cada um destes dois componentes.

Não se preocupe se esta distinção não faz muito sentido para você agora. As duas principais coisas que você precisa saber neste ponto são:

\begin{enumerate}
\item Tudo o que você digita no SuperCollider está na linguagem do SuperCollider (o cliente): é onde você escreve e executa comandos e vê os resultados na Post window.
\item Tudo o que faz som no SuperCollider está vindo do servidor---o “motor sonoro”, por assim dizer---, controlado por você através da linguagem do SuperCollider.
\end{enumerate}

\subsection{Iniciando o Servidor}
Seu programa “Olá Mundo” não produziu som algum: tudo aconteceu na linguagem e o servidor nem chegou a ser usado. O próximo exemplo fará som, então precisamos ter certeza que o Servidor está ligado e funcionando.

O jeito mais fácil de iniciar o servidor é com o atalho [ctrl+B]. Alternativamente, você pode clicar nos zeros da Barra de Status: um menu aparece e uma das opções é “Boot Server” (“Iniciar Servidor”). Você verá alguma atividade na Post Window enquanto o servidor está iniciando. Depois que você tiver iniciado o servidor com sucesso, os números na Barra de Status vão ficar verdes. Você terá de fazer isso todas as vezes que você iniciar o SC, mas apenas uma vez por sessão.


