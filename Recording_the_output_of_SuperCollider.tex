\section{Como gravar os sons do SuperCollider}

Logo você vai querer começar a gravar a saída de som dos seus patches de SuperCollider. Eis um jeito rápido:
 
\begin{lstlisting}[style=SuperCollider-IDE, basicstyle=\scttfamily\footnotesize]
// GRAVAÇÃO RÁPIDA
// Começar a gravar:
s.record;
// Faça algum som bacana:
{Saw.ar(LFNoise0.kr([2, 3]).range(100, 2000), LFPulse.kr([4, 5]) * 0.1)}.play;
// Pare de gravar:
s.stopRecording;
// Opcional: GUI com botão de gravação, controle de volume, botão de mudo.
s.makeWindow;
\end{lstlisting}
 
A Post window mostra o caminho para a pasta onde o arquivo foi salvo. Vá até o arquivo, abra-o no Audacity ou um programa similar e verifique se o arquivo foi mesmo gravado. Para mais informações, veja o arquivo de Ajuda de "Server" (role para baixo até "Recording Support"). Também online em \url{http://doc.sccode.org/Classes/Server.html}.
