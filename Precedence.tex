\section{Precedência}

O SuperCollider segue a ordem de precedência da esquerda para a direita, independente da operação. Isso significa, por exemplo, que multiplicação \emph{não} acontece primeiro:

\begin{lstlisting}[style=SuperCollider-IDE, basicstyle=\scttfamily\footnotesize]
// No colégio, o resultado era 9; no SC é 14:
5 + 2 * 2; 
//Use parênteses para forçar uma ordem de operações específica:
5 + (2 * 2); // igual a 9.
\end{lstlisting}

Quando se combina mensagens e operações binárias, mensagens assumem precedência. Por exemplo, em \texttt{5 + 2.squared}, a elevação ao quadrado acontece primeiro.
